\chapter*{Introduction}
\addcontentsline{toc}{chapter}{Introduction}
Each and every person on earth explores the world from the day of and continues to learn new things all his life. Education in the so-called developed world is essential for later employment.

@todo 

\section*{Khan Academy}
Khan Academy is an online tool providing free access to instructional videos and exercises covering various subjects including math, history, programming, economics, and more.

@todo

\paragraph{Screencast}
A screencast is a video created by recording computer screen output, often accompanied with an audio commentary. This process can be used in many different ways, for example to record a tutorial explaining how to use a specific computer program. The quality of the recorded video depends mainly on the resolution of the user's device resolution and the recorded area of the screen. Khan Academy uses screen capturing tools to record the virtual canvas of a bitmap editor, onto which the author draws using the tools of the editor and talks about the covered subject.

\section*{Vector Screencast project}
A year ago two members of ``Khanova škola'', the Czech brach of the Khan Academy, came with an idea of improving the current technical solution of recording and displayling educational videos on their website. It turns out that some of the videos recorded only a few years ago don't look very well or the hand-drawn text in these videos is not even legible on large displays. The other extreme are small displays of tablets and smartphones, where the downscaled letters are too small to read and can't be zoomed well in most video players. Their idea was to create a vector-based animation instead of classical bitmap-based screencasts. This animation could be scaled to any display resolution without any loss of information. The process of recording can be simulated in any distant future based on the recorded data of user's behaviour. The animation can be rendered using a different algorithm and the author would not have to record his screen again. As a result, the video will never become obsolete because of its poor visual quality.

One of the other reasons for this type of solution was the possible decrease of the size of data transfered over the Internet, as most of the image does not change between each two frames and very often the image doesn't change at all. The file size of a vector-based format does not correspond to the quality of the video. In regular bitmap formats, the higher the resolution of an image is, the the larger the data file is. Vector-based format has only one version for every display resolution.

\section*{Thesis structure}
@todo