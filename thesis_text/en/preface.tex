\chapter*{Introduction}
\addcontentsline{toc}{chapter}{Introduction}
A year ago two members of ``Khanova škola'', the Czech brach of the Khan Academy, came with an idea of improving the current technical solution of recording and displayling educational videos on their website. It turns out that some of the videos recorded only a few years ago don't look very well or the hand-drawn text in these videos is not even legible on large displays. The other extreme are small displays of tablets and smartphones, where the downscaled letters are too small to read and can't be zoomed well in most video players. Their idea was to create a vector-based animation instead of classical bitmap-based screencasts. This animation could be scaled to any display resolution without any loss of information. The process of recording can be simulated in any distant future based on the recorded data of user's behaviour. The animation can be rendered using a different algorithm and the author would not have to record his screen again. As a result, the video will never become obsolete because of its poor visual quality.

One of the other reasons for this type of solution was the possible decrease of the size of data transfered over the Internet, as most of the image does not change between each two frames and very often the image doesn't change at all. The file size of a vector-based format does not correspond to the quality of the video. In regular bitmap formats, the higher the resolution of an image is, the the larger the data file is. Vector-based format has only one version for every display resolution.

\section*{Khan Academy}
\label{sec:khan-academy}
The idea of \textit{Khan Academy} originated in 2003 when Salman Khan began tutoring his cousin over an instant messenger via drawing pictures with a computer mouse. Salman then started to record these videos and put them on his Youtube channel, so someone could watch them later. This channel became the basis of Khan Academy.

Khan Academy became known and has grown a lot, but the style of Khan Academy videos remained the same. A person draws lines and diagrams using a bitmap editor on his computer and talks about the subject aloud while recording his computer screen and recording his voice using a microphone. This style is sometimes called the ``Khan-style video'' (KSV)\nomenclature{KSV}{Khan-style video} \footnote{https://www.youtube.com/watch?v=Ohu-5sVux28}. These videos are then uploaded to Youtube and embedded in the Khan Academy website.

Apart from the video lectures, the website also contains exercises and quizzes to encourage students in learning. The pace of the lesson depends on the student. He can pause the videos or watch them multiple times before continuing with the lesson.

Most of the videos are recorded in English, but many of the videos are translated into other languages - by replacing the audio track with a different one or with subtitles. One of the projects working on the localization of Khan Academy videos is an official Czech branch called ``Khanova Škola''.

\subsection*{``Khanova Škola''}
\textit{Khanova Škola} is the Czech branch of the Khan Academy \cite{khanova_skola}. It is a non-profit organization. Volunteers around ``Khanova Škola'' translate the original videos from English to Czech to make them accessible for children at Czech elementary schools.

\subsection*{Screencast}
A screencast is a video created by recording computer screen output, often accompanied with an audio commentary. This process can be used in many different ways, for example to record a tutorial explaining how to use a specific computer program. The quality of the recorded video depends mainly on the resolution of the user's device resolution and the recorded area of the screen. Khan Academy uses screen capturing tools to record the virtual canvas of a bitmap editor, onto which the author draws using the tools of the editor and talks about the covered subject.

\section*{Goal of the thesis}
\addcontentsline{toc}{section}{Goal of the thesis}

The result of this thesis should be an open-source library suitable for extending any web application, like ``Khanova Škola'', with the abilities of recording and playing KSVs in all modern web browsers. Screencast will be recorded and played using a technology based on vector graphics to guarantee the quality of the video on any display and to achieve lower file size of the video file. An appropriate vector-based file format should be chosen or defined to store screencast data.

The library should be easily adjustable and configurable for different purposes. User interface should be fully translatable.

\section*{Thesis structure}
\addcontentsline{toc}{section}{Thesis structure}
The first part of the thesis is about distance education in general. Reader should get a notion of current distance education systems and their limits.

The technical requirements of the project are described in the following chapter. The third chapter is the analysis of the available tools and algorithms and which approaches to choose. A file format for the puroposes of this project is defined and described afterwards.

Following chapters contain information about the implementation of the library and its components, how the library can be integrated into websites, and how to use the library to record and play screencasts.