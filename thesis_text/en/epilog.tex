\chapter*{Conclusion}
\addcontentsline{toc}{chapter}{Conclusion}

The goal of this thesis was to create an open-source library, which will allow users to create educational websites for creating and playing KSVs. The library uses a vector-based approach. Video is rendered with respect to user's screen resolution and the size of the viewport.

The project of Vector Screencast is available online as a public Git repository\footnote{https://github.com/simonrozsival/vectorvideo}. The library is written in a programming language, which should be easy to learn for all web developers. The library uses an object oriented design and is very modular, which makes it easy to extend with a custom rendering method, file format support or user interface.

\paragraph{File format} Vector Screencast data is stored in a specificaly formed SVG document extended with a foreign namespace. The data contains information of author's cursor movement and applied pressure and prerendered lines. This type of animation cannot be viewed in any other video player, but a preview of the final state of the blackboard is displayed when the file is opened in a regular SVG viewer. Audio is recorded as an uncompressed WAV file. This file should be converted into one or more common binary audio formats, with respect to web browsers support of audio formats. 

\subparagraph{File size}

\section{Future work}
The library is ready and anyone can use it today in his website for creating and playing KSVs. There are tools for drawing lines of different widths and colors onto a virtual canvas, erasing parts of the canvas and clearing the whole canvas. Some authors might be missing some tools they use in their bitmap editors when recording KSV -- including bitmap images or photos, using a text tool or drawing straight lines. It would not be hard to implement these tools, but it would have to be thought over, whether they do not breach the idea of simple, natural look of KSV.

\paragraph{Binary file format}
SVG has proven to be a sufficient container for Vector Screencast data. A binary format should be used to achieve more data savings. Most of the data consists of coordinates and time values. Numbers are stored as strings in XML. Each digit character in an UTF-8 document takes up 1~B of data storage. Each component of a coordinate is a number with typical value in the range of 0 to 9999 with several decimal places -- the library allows three decimal places at maximum. A typical coordinate component value, like ``123.5'', takes up more than 4~B, with the decimal dot character included. Time values, which express time in miliseconds, consist of more than four digits as soon as they represent a time value of more than ten seconds. If these numbers were saved binary as a 32-bit integer, or 32-bit floating point number, it would take up only 4 bytes of memory in total while maintaining the same precission in most cases.

This project was designed to be independent on a specific file format. The library can be easily extended to support different file structures without editing its source code.

\subsection*{Known issues}
\paragraph{Audio recording} \textit{ScriptProcessorNode} interface is deprecated in W3C Editor's Draft of \textit{Web Audio API} from 21 June 2015 \cite{mic_deprecated}. This interface is used in this project and should be replaced using \textit{Audio Workers}, when the specification is finalized\cite{audio_workers}. 

\paragraph{Demo audio server} \textit{Audio recording server} program, which is included in the project as a demo, should be rewritten. This tool is not very robust and sometimes fails to save the data, it receives through a WebSocket, to a file on the server. Users must write this server program themselves at the moment.