\chapter{The Vector Video project}

In the spring of 2014 the people behind ``Khanova Škola'' --- the czech branch of the Khan Academy --- were looking for a person to develop an experimental technology of video based on vector graphics. Their idea was to record raw data of user's input and to render the lines on the target computer based on viewer's screen resolution and thus maintaining maximum quality. Thanks to the sparse nature of vectors in comparsion with bitmaps, the data consumption might also be reduced or at least similar. This animation would also be linked to an audio track of authors voice commentary forming a complex Khan Academy style video suitable for educational purposes.



\section{Video player requirements}
Any user should be able to play vector video in all major modern web browsers without any special software od plugin installed, including mobile browsers.

\section{Video recording tool requirements}




\section{Goals of this thesis}

The result of this thesis should be an open-source library suitable for extending any web application with the abilities of recording playling Khan Academy style videos in a web browser. An appropriate vector-based format should be chosen or defined to store video data.