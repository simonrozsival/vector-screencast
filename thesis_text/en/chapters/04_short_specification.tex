\chapter{The Vector Screencast project}

The overall project consists of two separate tools --- the \textit{recorder} and the \textit{player}. Each of these tools behaves differently and will be used by different people. While the video player will be used by general audience, the recorder will be used by a much narrower group of content creators.

Thanks to the sparse nature of vectors, in comparison to bitmaps, data consumption might also be reduced or at least similar. This animation would also be linked to an audio track of authors voice commentary forming a complex KSV suitable for educational purposes.

\section{Screencast recording tool requirements}
User, who wants to create a new screencast, should enter a website in his web browser and without installing any additional software start recording a video using his mouse, touchscreen or digital stylus and a microphone. This video should then be uploaded to a server.

If the user uses a pressure-sensitive stylus, then the applied pressure during drawing should be recorded too. The more pressure the user applies, the thicker the line will be and vice versa. Using this specific hardware might require additional drivers or specific software installed.

Different brush sizes, brush colors and background colors are available to the user. User must be able to erase certain parts of the canvas or the whole canvas at once. User must be able to pause and continue recording at any time.

Lines are immediately rendered on the screen as the user draws them, so he sees exactly the same output as the viewer will, when he is playing the video later. All the raw data collected from the user should be stored, including the data that have no effect for video playback, like recording cursor movement while recording is paused. This will allow the user to simulate the process of video recording and use it for further post-processing in any distant future. Recording of this extra information should be optional.

\section{Screencast player requirements}
Any user should be able to play vector video on-line in all major modern web browsers without any special software or plugin installed, including mobile browsers.

Video should be scaled appropriately to the size of the player and device's screen resolution. The lines should have the same shape they had as the author saw them while he was recording the screencast.

User can pause and continue with the playback of the video at any time. User can skip to any point of the video either forward or backward.

If the author of the video had recorded his voice, it will be played along with the video. Audio must be synchronized with the video whenever user plays or pauses the video or when he skips to a different point of the timeline.

User interface should be intuitive and easy to use either on a desktop computer using mouse and keyboard, but also with touchscreens on mobile devices.