\chapter{Using Vector Screencast library}
\label{ch:integration}

Using Vector Screencast library is intended to be as simple as possible. All the user needs to do is include a JavaScript file of the library and a CSS file of a theme into his HTML code and configure the player in a few lines of code.

\section{Obtaining the Vector Screencast library}
Prepared library files can be obtained either from the attached files or from Git repository https://github.com/simonrozsival/vectorvideo in the \verb|/release/VectorScreencast| and \verb|/release/themes/| folders. You only need to copy files \verb|vector-screencast.min.js| and \verb|theme-default.min.css| into your project.

The library files must be linked to your document and you should make sure that your website will be displayed properly on mobile devices by specifying the \verb|viewport| meta tag in the \verb|head| section of your document \cite{html_viewport}. Also create an empty element with a specific \verb|id| attribute -- this will be the container, into which either the screencast player or the recorder will be placed.

An example of a HTML5 template with correct setup can look similarly (irrelevant parts of the document were let omitted and replaced by suspension points):

\begin{lstlisting}
<!DOCTYPE html>
<html>
	<head>
		...
		<meta name="viewport" content="width=device-width,initial-scale=1">
		<link rel="stylesheet" type="text/css" href="/path/to/theme-default.min.css" media="screen, projection">
		...
	</head>
	<body>
		...
		<div id="some-specific-id"></div>
		...
		<script src="/path/to/vector-screencast.min.js"></script>
	</body>
</html>
\end{lstlisting}

The initialisation scripts must be executed afte all web page resources are downloaded and the DOM is ready -- this can be achieved by putting this code inside a handler of the \verb|window.onload| event.

\section{Vector Screencast Player}
Inside your scripts, create a new instance of \verb|VectorScreencast.Player|. The constructor takes two arguments, first of them is the \verb|id| attribute of a container element and the second is a configuration object. The only obligatory property of the configuration object is the \verb|Source| property -- the URL of the source SVG file. 

There are several other interesting optional settings, that will help you customize the screencast player. One of them is the \verb|Localization| property, which takes an object implementing the \verb|VectorScreencast.Localization.PlayerLocalization| interface. To see the complete list of all configuration options and further details, please refer to the \verb|VectorScreencast.Settings.PlayerSettings| interface in the API reference of the project in the \verb|/docs/| folder of the attached files. You can see an advanced example of \verb|VectorScreencast.Player| usage \verb|/demo/public/play.html| in the attached files.

\section{Vector Screencast Recorder}

Inside your scripts, create a new instance of \verb|VectorScreencast.Player|. The constructor takes two arguments, first of them is the \verb|id| attribute of a container element and the second is a configuration object. The only obligatory property of the configuration object is the \verb|Source| property -- the URL of the source SVG file. 

There are several other interesting optional settings, that will help you customize the screencast player. One of them is the \textit{Localization} property, which takes an object implementing the \verb|VectorScreencast.Localization.PlayerLocalization| interface. To see the complete list of all configuration options and further details, please refer to the \verb|VectorScreencast.Settings.PlayerSettings| interface in the API reference of the project in the \verb|/docs/| folder of the attached files. You can see an advanced example of \verb|VectorScreencast.Player| usage \verb|/demo/public/play.html| in the attached files.

\paragraph{Audio recording server process}
@todo


\section{Embedding Vector Screencast Player}
To allow users embed videos from your website on their websites, create a HTML page, where the player will stretch over the whole screen. Then generate an HTML snippet with an \verb|<iframe>| pointing to your player in the \verb|src| attribute.

\section{Custom theme}
You can customize the look of the player and the recorder to match the design of your website or to change the layout of the controls. Use a custom CSS style sheet to override the default style.

You may want to start by editing the default style of the Vector Screencast. The source files are located in the \verb|/src/Themes/default| directory. These files are transpiled using the \textit{Less} CSS preprocessor. You can create a new theme by just modifying values of variables in the \verb|/src/Themes/default/variables/variables.less| file.

After you have created your CSS theme, use HTML \verb|<link>| tag to link the CSS stylesheet to your website.

\section{Demo project}
To make life easier to you, we have prepared an example project, where you can see the library in action. This project shows the use of
\begin{itemize}
	\item recording tool
	\item the player
	\item embedding the player
	\item demo HTTP server based on Node.js and Express.js
	\item demo audio recording server based on Node.js
\end{itemize}

To try the example project, follow section~\ref{sec:building_the_library} on page~\pageref{sec:building_the_library} and build the demo project by running \verb|gulp demo| command.

When the project is built, start the HTTP server and the audio recording server. To do that, open your shell console and change your working directory to \verb|/demo|. Here, start the servers by executing

\begin{lstlisting}
node server.js 3000 &
node audio.js http://localhost:3000 4000 &
\end{lstlisting}

This will start the local HTTP server listening on port 3000 and a local audio recording server listening on port 4000. If you need to change port numbers, do not forget to change the port number of the audio recording server also in \verb|/demo/public/record.html|. Both servers will run in background, but will keep printing log messages into your console window.

Open your web browser and go to \verb|http://localhost:3000|. This will open the main page of the demo project. You can then record videos, upload and play them and embed them into other local HTML files.

\paragraph{Online demo}
If you do not want to deploy the demo project at your own computer, visit \verb|http://www.rozsival.com| to try the demo online.