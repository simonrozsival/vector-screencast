\chapter{Integration of the library in an HTML web page}
\label{ch:integration}

Using Vector Screencast library is intended to be as simple as possible. All the user needs to do is include a JavaScript file of the library and a CSS file of a theme into his HTML code and configure the player in a few lines of code.

\section{Obtaining the Vector Screencast library}
Prepared library files can be obtained either from the attached files or from the GIT repository https://github.com/simonrozsival/vectorvideo in the \textit{/release/VectorScreencast} and \textit{/release/themes/} folders. You only need to copy files \textit{vector-screencast.min.js} and \textit{theme-default.min.css} into your project.

The library files must be linked to your document and you should make sure that your website will be displayed properly on mobile devices by specifying the \textit{viewport} meta tag in the \textit{head} section of your document\footnote{https://developers.google.com/speed/docs/insights/ConfigureViewport}. Also create an empty element with a specific \textit{id} attribute -- this will be the container, into which either the screencast player or the recorder will be placed.

An example of a HTML5 template with correct setup can look similarly (irrelevant parts of the document were let omitted and replaced by suspension points):

\begin{lstlisting}
<!DOCTYPE html>
<html>
	<head>
		...
		<meta name="viewport" content="width=device-width,initial-scale=1">
		<link rel="stylesheet" type="text/css" href="/path/to/theme-default.min.css" media="screen, projection">
		...
	</head>
	<body>
		...
		<div id="some-specific-id"></div>
		...
		<script src="/path/to/vector-screencast.min.js"></script>
	</body>
</html>
\end{lstlisting}

The initialisation scripts must be executed afte all web page resources are downloaded and the DOM is ready -- this can be achieved by putting this code inside a handler of the \textit{window.onload} event.

\section{Vector Screencast Player}
Inside your scripts, create a new instance of \textit{VectorScreencast.Player}. The constructor takes two arguments, first of them is the \textit{id} attribute of a container element and the second is a configuration object. The only obligatory property of the configuration object is the \textit{Source} property -- the URL of the source SVG file. 

There are several other interesting optional settings, that will help you customize the screencast player. One of them is the \textit{Localization} property, which takes an object implementing the \textit{VectorScreencast.Localization.PlayerLocalization} interface. To see the complete list of all configuration options and further details, please refer to the \textit{VectorScreencast.Settings.PlayerSettings} interface in the API reference of the project in the \textit{/docs/} folder of the attached files. You can see an advanced example of \textit{VectorScreencast.Player} usage \textit{/demo/public/play.html} in the attached files.

\section{Vector Screencast Recorder}

Inside your scripts, create a new instance of \textit{VectorScreencast.Player}. The constructor takes two arguments, first of them is the \textit{id} attribute of a container element and the second is a configuration object. The only obligatory property of the configuration object is the \textit{Source} property -- the URL of the source SVG file. 

There are several other interesting optional settings, that will help you customize the screencast player. One of them is the \textit{Localization} property, which takes an object implementing the \textit{VectorScreencast.Localization.PlayerLocalization} interface. To see the complete list of all configuration options and further details, please refer to the \textit{VectorScreencast.Settings.PlayerSettings} interface in the API reference of the project in the \textit{/docs/} folder of the attached files. You can see an advanced example of \textit{VectorScreencast.Player} usage \textit{/demo/public/play.html} in the attached files.


\section{Embedding Vector Screencast Player}
@todo

\section{Custom theme}
@todo