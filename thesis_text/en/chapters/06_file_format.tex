\chapter{Vector Screencast file format}

Vector video file must contain all the information about cursor movement and precise data of the lines, including their variable width, and the times at which every single segment of a line is drawn. This data is then used to reconstruct the precise movement of the author's cursor and draw the very sqme lines at the very same pace.

There are several vector based formats available.

\paragraph{Scalable Vector Graphics (SVG)}
Scalable Vector Graphics (SVG)\footnote{http://www.w3.org/TR/SVG/} is an Extensible Markup Language (XML)\footnote{http://www.w3.org/XML/} based file format designed for describing two-dimensional vector images. It is an open format developed and maintained by the W3C SVG Working Group \cite{}. Current W3C Recommendation is SVG 1.1 (Second Edition).

A valid SVG document must have an \textit{svg} root element with specific namespace attributes and specified \textit{width} and \textit{height} attributes. An example of an empty, but valid, SVG document might look as shown:

\begin{verbatim}
<?xml version="1.0"?>
<svg version="1.1"
        width="470"
        height="100"
        xmlns="http://www.w3.org/2000/svg"
        xmlns:xlink="http://www.w3.org/1999/xlink"
        xmlns:ev="http://www.w3.org/2001/xml-events">

</svg>
\end{verbatim}

The specification of SVG introduces several graphical primitives, that can be used to compose complex shapes. Each primitive is represented by an XML element and a set of attributes. The list of primitives is long, though we will need only two of them -- the \verb|<circle>|, \verb|<rect>| and \verb|<path>|. The \verb|<g>| element\footnote{http://www.w3.org/TR/SVG/struct.html\#GElement} is intended for grouping related graphics elements. These groups might be also nested.