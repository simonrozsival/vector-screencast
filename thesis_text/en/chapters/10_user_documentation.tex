\chapter{Users' documentation}
The Video Screencast tools were designed to be easy to use and allow users to work with them right away. The following sections summarize the way user can interact with the default UI of both library tools. UI might be changed by the user of the library and the user experience might change. It is then up to the user of the library to explain the specifics of the new interface to the screencast authors and audience.

\section{Vector Screencast Player}
Once the screencast is ready to be played, you can start learning. You should see a simillar interface to the one shown at figure~\ref{fig:player}. The area of the player consists of two main parts -- the blackboard and control bar.

\begin{figure}
	\centering
		\includegraphics[width=150mm]{../img/player_screenshot.eps}
		\label{fig:player}
		\caption{The default UI of Vector Screencast Player}
\end{figure}

\paragraph{Play/Pause/Replay}
In the bottom left part of the screencast player UI there is a button with a ``play'' icon. You start or resume the playback by clicking on the button.

When the screencast is playing, the icon changes to a ``pause'' icon. By clicking on the same button, playback of the screencast is paused. The icon will then change to the ``play'' icon.

When the end is reached, the playback stops and the icon changes to a ``replay'' icon. When you click on the button now, playback will start over from the very beginning.

Instead of clicking on the button, you may press the spacebar key on your keyboard, if you have one. You can also click or tap the blackboard to play/pause playback.

\paragraph{Skipping parts of the screencast}
In the top section of the control bar there is a timeline. It shows progress of the video. You can click on the timeline to change current position of the video to the one corresponding to your selection. If you hover your mouse over the timeline, a box with time information of that point will appear. On touch some devices, you might have to tap second time to confirm the selected position.

Instead of clicking on the timeline, you can press left and right arrow keys on your keyboard. Each stroke of the keys will skip five seconds of the screencast forwards or backwards.

\paragraph{Control bar hiding}
When the screencast is playing, the control bar is not needed and could be hidden. To turn hiding on and off, press the right most button in the control bar. You probably would not see the effect immediatelly. The control bar hides only if the video is being played. It is shown automatically when the video is paused or reaches end. It is shown also whenever you hover your mouse over the remaining visible parts of the control bar.

\paragraph{Audio controls}
You can control audio volume by clicking on buttons in the ``Volume controls'' section of the control bar. Volume can be decreased, increased or muted/unmuted.


\section{Vector Screencast Recorder}
After you enter the page with the recorder, you should see a simillar interface to the one shown at figure~\ref{fig:recorder}. The area of the recording tool consists of two main parts -- the blackboard and control bar.

\begin{figure}
	\centering
		\includegraphics[width=150mm]{../img/recorder_screenshot.eps}
		\label{fig:recorder}
		\caption{The default UI of Vector Screencast Recorder}
\end{figure}

\paragraph{Microphone access and audio recording}
Your browser will ask you to allow access to your microphone. Confirming this prompt is neccessary for audio recording. If this message does not show, your computer probably does not have a microphone, your browser does not support audio recording or you have denied access to your microphone earlier. 

A white not crossed out icon of a microphone should appear in the ``Audio recording'' section of the control panel. If the icon is red, audio recording is not available. Please make sure you use the latest version of your web browser and check you browser and operating system microphone settings if you encounter problems.

If you do not want to record audio or you want to mute the microphone in the middle of recording, press the microphone button. You will resume recording your voice by clicking the button one more time.

\paragraph{Drawing}
You can start drawing as soon as audio recording is set up. You can now prepare the inital look of the screen. When you click the ``Start'' button, your cursor movement and voice will start being recorded. You can pause and resume recording at any time and as many times as you want.

Before you start drawing a line, select brush size and brush color from the palette in the control bar. Draw a line by pressing your mouse, stylus or touching your touchscreen and moving the cursor around the blackboard. End a line by releasing mouse button, stylus pressure of lifting your finger.

It is recommended to use a pressure-sensitive graphics tablet to achieve best results. If you use a touchscreen, consider using a specialized stylus instead of fingers to create better drawings.

\paragraph{Erasing}
If you want to clear the whole blackboard, select a color from the color palette and click on the button in the ``Erase everything'' section of the control bar. The color of the button should correspond to your previousely selected color. Everything will be removed from the balckboard and the background color of the ``\textit{black}board'' will be set to the selected color.

\paragraph{Saving the recording}
After you have finished recording and the recording is stopped, press the ``Upload'' button. The data will be uploaded to the server. This process will take some time, even a few minutes, depending on your Internet connection uplink speed and the length of your recording. After the video is uploaded, you will be informed
